\chapter{Fragmentos de c�digo}
\label{Anexo:codigo}

A continuaci�n se muestran los fragmentos de c�digo m�s relevantes de cada uno de los componentes del sistema, con el fin de ayudar al lector a comprender las decisiones de implementaci�n.

\lsection{Cliente}

%\lstinputlisting[language=XML, firstline=1, lastline=106]{../01-CardboardApps/SurfaceViewCarboard/app/src/main/res/layout/form.xml}

\subsection{AddressManager.java}
%\todo[inline]{Anexo c�digo: \currentname}

\begin{figure}[H]
	\inputminted[firstline=7,lastline=12]{java}{\fpandroidclient/app/src/main/java/com/example/borch/surfaceviewcarboard/AddressManager.java}
	\caption{\textbf{Patr�n de dise�o \textit{Singleton}}. La �nica instancia de esta clase puede ser referenciada invocando al m�todo p�blico y est�tico \textit{getInstance()}}
	\label{fig:addressmanager_singleton}
\end{figure}

\begin{figure}[H]
	\inputminted[firstline=13,lastline=18]{java}{\fpandroidclient/app/src/main/java/com/example/borch/surfaceviewcarboard/AddressManager.java}
	\caption{\textbf{Atributos de la instancia}. Estos son los datos que se almacenan en esta clase. Sus correspondientes \textit{getters} y \textit{setters} son de acceso p�blico}
	\label{fig:addressmanager_atributos}
\end{figure}

\subsection{PositionSender.java}
%\todo[inline]{Anexo c�digo: \currentname}

\begin{figure}[H]
	\inputminted[firstline=17,lastline=24]{java}{\fpandroidclient/app/src/main/java/com/example/borch/surfaceviewcarboard/PositionSender.java}
	\caption{\textbf{Atributos de la clase}. }
	\label{fig:positionsender_atributos}
\end{figure}

\begin{figure}[H]
	\inputminted[firstline=27,lastline=33]{java}{\fpandroidclient/app/src/main/java/com/example/borch/surfaceviewcarboard/PositionSender.java}
	\caption{\textbf{M�todo constructor}. }
	\label{fig:positionsender_constructor}
\end{figure}

\begin{figure}[H]
	\inputminted[firstline=36,lastline=41]{java}{\fpandroidclient/app/src/main/java/com/example/borch/surfaceviewcarboard/PositionSender.java}
	\caption{\textbf{M�todo \textit{send()}}. Este m�todo se encarga de enviar un mensaje, en forma de una cadena de caracteres, a trav�s del socket UDP previamente inicializado en el constructor.}
	\label{fig:positionsender_send}
\end{figure}


\subsection{FormActivity.java}
%\todo[inline]{Anexo c�digo: \currentname}

\begin{figure}[H]
	\inputminted[firstline=12,lastline=22]{java}{\fpandroidclient/app/src/main/java/com/example/borch/surfaceviewcarboard/FormActivity.java}
	\caption{\textbf{Definici�n de la clase, AddressManager y \textit{onCreate()}}. }
	\label{fig:formactivity_clase}
\end{figure}

\begin{figure}[H]
	\inputminted[firstline=24,lastline=49]{java}{\fpandroidclient/app/src/main/java/com/example/borch/surfaceviewcarboard/FormActivity.java}
	\caption{\textbf{M�todo \textit{updateValues()}}. Controlador del bot�n \enquote{Update values}. Accede a los \textit{EditText} de la vista y los almacena en el modelo (AddressManager)}
	\label{fig:formactivity_updatevalues}
\end{figure}

\begin{figure}[H]
	\inputminted[firstline=52,lastline=57]{java}{\fpandroidclient/app/src/main/java/com/example/borch/surfaceviewcarboard/FormActivity.java}
	\caption{\textbf{M�todo \textit{nextActivity()}}. }
	\label{fig:formactivity_nextactivity}
\end{figure}


\newpage

\subsection{MainActivity.java}

\begin{figure}[h]
	\inputminted[firstline=91,lastline=110]{java}{\fpandroidclient/app/src/main/java/com/example/borch/surfaceviewcarboard/MainActivity.java}
	\caption{\textbf{Inicializaci�n de objeto VideoView}. El mismo proceso se realiza para myVideoViewRight y ambos comparten el mismo MediaController.}
	\label{fig:mainactivity_videoviewinit}
\end{figure}

\begin{figure}[h]
	\inputminted[firstline=133,lastline=142]{java}{\fpandroidclient/app/src/main/java/com/example/borch/surfaceviewcarboard/MainActivity.java}
	\caption{\textbf{Acceso a los sensores y creaci�n del objeto \textit{PositionSender}}: }
	\label{fig:mainactivity_setupsensorposition}
\end{figure}

\begin{figure}[h]
	\inputminted[firstline=161,lastline=174]{java}{\fpandroidclient/app/src/main/java/com/example/borch/surfaceviewcarboard/MainActivity.java}
	\caption{\textbf{Hilo de env�o} de la posici�n al servidor de control}
	\label{fig:mainactivity_hilosender}
\end{figure}

\begin{figure}[H]
	\inputminted[firstline=181,lastline=212]{java}{\fpandroidclient/app/src/main/java/com/example/borch/surfaceviewcarboard/MainActivity.java}
	\caption{\textbf{M�todo \textit{onSensorChanged()}}: Realiza el c�lculo de las coordenadas polares (figura~\ref{fig:coordenadas_polares}) cada vez que los sensores cambian de valor.}
	\label{fig:mainactivity_onsensorchanged}
\end{figure}

\begin{figure}[H]
	\inputminted[firstline=278,lastline=285]{java}{\fpandroidclient/app/src/main/java/com/example/borch/surfaceviewcarboard/MainActivity.java}
	\caption{\textbf{M�todo \textit{onBackPressed()}}: se ejecuta cuando se presiona el bot�n de los dispositivos Android con forma de tri�ngulo.}
	\label{fig:mainactivity_onbackpressed}
\end{figure}


\lsection{Servidores de v�deo}

\subsection{deploy.sh}
\begin{figure}[H]
	\inputminted[firstline=1,lastline=11]{bash}{\fpservidoresvideo/deploy.sh}
	\caption{\textbf{Script de despliegue} de los servidores multimedia.}
	\label{fig:despliegue_servidores_video}
\end{figure}	

\newpage

\lsection{Servidor de control}

\subsection{servo\_serial\_read.ino}
\begin{figure}[h]
	\inputminted[firstline=1,lastline=8]{cpp}{\fparduinocode/servo_serial_write/servo_serial_write.ino}
	\caption{\textbf{Cabecera del c�digo del Arduino}.}
	\label{fig:header_arduino}
\end{figure}	


\begin{figure}[H]
	\inputminted[firstline=9,lastline=27]{cpp}{\fparduinocode/servo_serial_write/servo_serial_write.ino}
	\caption{\textbf{Funci�n auxiliar \textit{myRead()}}. Lee caracter a caracter del puerto serial hasta encontrar el valor '\textbackslash n' y convierte la cadena de caracteres obtenida en un valor num�rico.}
	\label{fig:my_read_arduino}
\end{figure}	
%

\begin{figure}[H]
	\inputminted[firstline=27,lastline=34]{cpp}{\fparduinocode/servo_serial_write/servo_serial_write.ino}
	\caption{\textbf{Funci�n \textit{setup()} de Arduino}. Esta funci�n se ejecuta al arrancarse la placa.}
	\label{fig:setup_arduino}
\end{figure}	


\begin{figure}[H]
	\inputminted[firstline=36,lastline=43]{cpp}{\fparduinocode/servo_serial_write/servo_serial_write.ino}
	\caption{\textbf{Funci�n \textit{loop()} de Arduino}. Esta funci�n se ejecuta indefinidamente despu�s de ejecutarse \textit{setup()}.}
	\label{fig:loop_arduino}
\end{figure}	

\subsection{control\_servo.py}
%\todo[inline]{Anexo c�digo: \currentname}
\begin{figure}[H]
	\inputminted[firstline=28,lastline=34]{python}{\fpservidorcontrol/control_servo.py}
	\caption{\textbf{Definici�n de la clase y constructor de \textit{ServoControl}}. El �ngulo inicial es $\Psi_0 = 0^\circ$. El constructor crea un objeto \textit{Serial} en el puerto 9600 a partir de su ruta en el sistema operativo.}
	\label{fig:controlservo_clase}
\end{figure}

\begin{figure}[H]
	\inputminted[firstline=62,lastline=83]{python}{\fpservidorcontrol/control_servo.py}
	\caption{\textbf{Definici�n de la clase y constructor de \textit{ServoControl}}.}
	\label{fig:controlservo_moveangle}
\end{figure}

\subsection{udp\_server.py}

\begin{figure}[H]
	\inputminted[firstline=7,lastline=16]{python}{\fpservidorcontrol/udp_server.py}
	\caption{\textbf{Inicializaci�n de recursos del servidor de control}.}
	\label{fig:setup_udp_server}
\end{figure}

\begin{figure}[H]
	\inputminted[firstline=17,lastline=28]{python}{\fpservidorcontrol/udp_server.py}
	\caption{\textbf{Bucle del servidor de control}.}
	\label{fig:loop_udp_server}
\end{figure}


%\begin{listing}[ht]
%\inputminted{java}{../01-CardboardApps/SurfaceViewCarboard/app/src/main/java/com/example/borch/surfaceviewcarboard/MainActivity.java}
%\caption{Example from external file}
%\label{listing:3}
%\end{listing}

\newpage \thispagestyle{empty} % P�gina vac�a 