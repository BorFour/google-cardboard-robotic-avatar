\chapter{Estado del arte}
\label{chap:estadodelarte}
 
\lsection{Introducci�n}
\todo[inline]{Estado del arte: \currentname}
La realidad virtual es un concepto que existe desde hace d�cadas, aunque durante gran
parte de este tiempo se ha visto como mera ciencia ficci�n. Actualmente, ya existen dispositivos dise�ados espec�ficamente para que el usuario pueda visualizar mundos virtuales en 3D. 

En primavera de 2016 se concentran las fechas en las que numerosas empresas prometieron
comercializar productos con esta funcionalidad, como son Oculus Rift, HTC Vive y PlayStation VR entre otros. 

Por otra parte, la realidad aumentada combina en tiempo real la visi�n del entorno f�sico
con elementos virtuales que a�aden informaci�n a lo que uno podr�a ver simplemente con sus ojos. Microsoft Hololens llegar� al mercado tambi�n en la primera mitad del a�o 2016.
El producto descrito a continuaci�n no se podr�a clasificar como ninguno de estos dos
anteriores, aunque s� que es cierto que est� fuertemente ligado a estos dos conceptos.
En septiembre de 2015, Snapchat incorpora la realidad aumentada a su aplicaci�n, pudiendo personalizar fotograf�as y v�deos en tiempo real con distinto contenido basado en el reconocimiento facial.

%\lsection{Definici�n.} \label{sec:definicion}

\lsection{Historia, nacimiento y evoluci�n.} \label{sec:historia}
\todo[inline]{Estado del arte: \currentname}


\lsection{Estado actual}  \label{sec:estadoactual}
\todo[inline]{Estado del arte: \currentname}

\lsection{Conceptos previos} \label{sec:conceptosprevios}
\todo[inline]{Estado del arte: \currentname}
\newpage \thispagestyle{empty} % P�gina vac�a 