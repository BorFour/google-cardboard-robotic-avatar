\chapter{Conclusiones y trabajo futuro}
\label{chap:conclusiones}


\lsection{Conclusiones}
\todo[inline]{\currentname}

Los objetivos principales de este proyecto se han visto realizados.

Personalmente, se ha llevado a cabo el desarrollo de un sistema distribuido heter�geneo, integrando tecnolog�as punteras y variadas como culmen de mis estudios en \carrera. 

El mundo de las redes multimedia y la transmisi�n de v�deo y audio tiene muchos detalles. A pesar de que no ha sido demasiado complejo integrar estas herramientas para desplegar un servidor de v�deo. Las implementaciones de los protocolos a nivel de aplicaci�n de transmisi�n de multimedia (RTSP en este caso) son complejas a bajo nivel, as� como los c�decs, los formatos y todos sus par�metros, variables y metadatos.
		
Por otro lado y como se ha comentado previamente, la realidad virtual se encuentar en auge, motivo por el cual han surgido y surgen nuevas librer�as, herramientas, frameworks, etc. orientados la creaci�n de programas y aplicaciones de VR. Al ser tan nuevas, todav�a necesitan madurar arreglando fallos y bugs, documentando mejor las APIs e implementando m�s funcionalidad necesaria y/o �til para el desarrollo de este tipo de software.



\lsection{Trabajo futuro}
\todo[inline]{\currentname}
\begin{itemize}

	\item {
		Tareas de optimizaci�n de la QoS, a saber:
		\begin{enumerate}
			\item Desarrollar una \textbf{vista} para Android (android.view.View) \textbf{espec�fica para streams RTSP}, que ofrezca mayores prestaciones tales como menor delay, mantener la sesi�n RTSP, sincronizaci�n de los dos flujos de v�deo, etc.
		\end{enumerate}
	}	
	\item Utilizar la API Google VR for Android *REFERENCIA*. Hasta hace unos meses, se llamaba Cardboard API. Actualmente tambi�n incluye soporte para DayDream VR\cite{online:daydream}, cuyo lanzamiento tendr� lugar en oto�o del 2016, y est� m�s documentada, con m�s ejemplos y m�s funcionalidad. Los v�deos se mostrar�an sobre una textura de OpenGL, que har�an que los v�deos se mostrasen como en la figura~\ref{fig:cardboard_3d}
	\item Implementar la rotaci�n sobre los ejes X e Y (Pitch y Roll respectivamente, figura ~\ref{fig:coordenadas_polares}).
	\item \textbf{Utilizar motores paso a paso o \textit{steppers}} para mover la estructura sobre la que se encuentren las c�maras.
	\item A�adir un stream m�s que se corresponda con el audio, grabado en el servidor a trav�s de un micr�fono cualquiera.
	
\end{itemize}
\newpage \thispagestyle{empty} % P�gina vac�a 