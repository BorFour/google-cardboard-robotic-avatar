\chapter*{Resumen}

\section*{Resumen}
%\todo[inline]{ \currentname}

El objetivo principal de este Trabajo de Fin de Grado es desarrollar un sistema distribuido heterog�neo, combinando visi�n remota y realidad virtual, donde el usuario pueda ver a trav�s de los ojos de un \textit{avatar rob�tico}.

El cliente consiste en unas gafas de realidad virtual, como lo son Google Cardboard. Es una artilugio de cart�n con dos huecos para un par de lentes, que apuntan a una cavidad donde se ha de situar un smarthphone. Una aplicaci�n mostrar� dos im�genes en la pantalla del dispositivo m�vil que ser�n observadas a trav�s de estas lentes por el usuario.

En la parte del servidor tenemos un par de videoc�maras situadas en una estructura sobre un motor. Una parte del servidor env�a los fotogramas capturados por las c�maras a trav�s de un protocolo de multimedia en red hasta el cliente, donde cada transmisi�n se corresponde con un ojo del usuario.

Por otra parte, el servidor tambi�n se encargar� de hacer rotar la estructura con las c�maras, controlando el motor seg�n los movimientos de la cabeza del usuario. La aplicaci�n env�a peri�dicamente la rotaci�n del dispositivo, que se calcula utilizando el aceler�metro y la br�jula del smartphone, a este servidor de control.






\section*{Palabras Clave}
%\todo[inline]{ \currentname}
Google Cardboard, Realidad Virtual, Redes Multimedia, Android, Servomotor, App, Transmisi�n de v�deo, Tiempo Real, Sensores, Sistema Distribuido

\newpage

%-------------------------------------------------------------------------------------------------------------------------------------
\section*{Abstract}
%\todo[inline]{ \currentname}
The main goal of this Bachelor Thesis is to develop an heterogeneous distributed system, combining remote vision and virtual reality, where the user can look through the eyes of a \textit{robotic avatar}.

The client consists of a pair of virtual reality glasses, such as Google Cardboard. It is a cardboard made gadget with two holes for a pair of lens that look to a cavity where the smartphone must be placed. An application will show two images on the mobile device's screen that will be observed through this lens by the user.

There is a couple of videocameras on the server side, assembled in a structure on a motor. A part of the server sends the frames captured by the cameras using a multimedia networking protocol to the client, where each stream corresponds to an eye.

On the other hand, the server is also responsible for rotating the structure with the cameras controlling the motor according to the movements of the user's head. The application periodically sends the rotation of the device, which is calculated using the smartphone's accelerometer and compass, to the control server.

\section*{Key words}
%\todo[inline]{ \currentname}
Google Cardboard, Virtual Reality, Multimedia Networking, Android, Servomotor, App, Video Streaming, Real Time, Sensors, Distributed System