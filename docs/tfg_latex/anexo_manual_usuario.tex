\chapter{Manual de usuario}
\label{Anexo:manual_usuario}

Manual de ayuda al usuario para acceder al c�digo y poder hacer funcionar el sistema.

\lsection{Repositorio}
%\todo[inline]{Manual de usuario: \currentname}

\subsection{C�mo clonar el repositorio}

En la terminal de Linux, ejecutar el siguiente comando:

\begin{minted}{bash}
mkdir <directorio_destino>
cd <directorio_destino>
git clone https://BorFour@bitbucket.org/BorFour/googlecardboardroboticavatar.git
\end{minted}

\lsection{Dependencias}
\begin{itemize}
	\item Sistema operativo basado en UNIX
	\item Git
	\item VLC
	\item Python 2.7
	\item Arduino IDE	
\end{itemize}
\newpage
\lsection{Uso}
\subsection{C�digo arduino}
Para compilar y subir a la placa el c�digo del arduino, deben seguirse los siguientes pasos:
\begin{enumerate}
	\item Conectar el Arduino al PC con el cable USB
	\item Abrir el c�digo. File --> Open... --> 
	
	<ruta del proyecto>/src/
	ArduinoCode/servo\_serial\_write/servo\_serial\_write.ino
	\item Importar la librer�a VarSpeedServo. Sketch --> Import library... --> Add library... --> <ruta del proyecto>/src/ArduinoCode/VarSpeedServo.zip
	\item Compilar y subir el c�digo. Ctrl + U.
\end{enumerate}

\subsection{Despliegue de los servidores}
En el directorio ra�z, ejecutar el siguiente comando en la terminal de Linux:

\begin{minted}{bash}
	make
\end{minted}

Esto har� que se ejecuten los dos siguientes scripts:

\begin{itemize}	
	\item \textbf{Despliegue de los servidores de v�deo}: el script src/VideoServers/deploy.sh despliega los servidores de v�deo en los puertos 8554 y 8556 (c�maras izquierda y c�mara derecha respectivamente)
	\item \textbf{Despliegue del servidor de control}: despliega el servidor UDP en el puerto 8558, asioc�ndolo a la terminal en la que se ejecute el script control\_server.sh	
\end{itemize}

\subsection{Instalaci�n de la aplicaci�n}

Para instalar la aplicaci�n en un smartphone con Android, seguir estos dos pasos:

\begin{itemize}
	\item Enviar el archivo gcra\_client.apk al dispositivo m�vil.
	\item Al abrir el archivo, Android se encargar� de instalarlo.
\end{itemize}

\newpage \thispagestyle{empty} % P�gina vac�a 